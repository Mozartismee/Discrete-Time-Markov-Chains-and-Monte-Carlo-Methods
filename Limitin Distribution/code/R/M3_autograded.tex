\documentclass[11pt]{article}

    \usepackage[breakable]{tcolorbox}
    \usepackage{parskip} % Stop auto-indenting (to mimic markdown behaviour)
    

    % Basic figure setup, for now with no caption control since it's done
    % automatically by Pandoc (which extracts ![](path) syntax from Markdown).
    \usepackage{graphicx}
    % Maintain compatibility with old templates. Remove in nbconvert 6.0
    \let\Oldincludegraphics\includegraphics
    % Ensure that by default, figures have no caption (until we provide a
    % proper Figure object with a Caption API and a way to capture that
    % in the conversion process - todo).
    \usepackage{caption}
    \DeclareCaptionFormat{nocaption}{}
    \captionsetup{format=nocaption,aboveskip=0pt,belowskip=0pt}

    \usepackage{float}
    \floatplacement{figure}{H} % forces figures to be placed at the correct location
    \usepackage{xcolor} % Allow colors to be defined
    \usepackage{enumerate} % Needed for markdown enumerations to work
    \usepackage{geometry} % Used to adjust the document margins
    \usepackage{amsmath} % Equations
    \usepackage{amssymb} % Equations
    \usepackage{textcomp} % defines textquotesingle
    % Hack from http://tex.stackexchange.com/a/47451/13684:
    \AtBeginDocument{%
        \def\PYZsq{\textquotesingle}% Upright quotes in Pygmentized code
    }
    \usepackage{upquote} % Upright quotes for verbatim code
    \usepackage{eurosym} % defines \euro

    \usepackage{iftex}
    \ifPDFTeX
        \usepackage[T1]{fontenc}
        \IfFileExists{alphabeta.sty}{
              \usepackage{alphabeta}
          }{
              \usepackage[mathletters]{ucs}
              \usepackage[utf8x]{inputenc}
          }
    \else
        \usepackage{fontspec}
        \usepackage{unicode-math}
    \fi

    \usepackage{fancyvrb} % verbatim replacement that allows latex
    \usepackage{grffile} % extends the file name processing of package graphics
                         % to support a larger range
    \makeatletter % fix for old versions of grffile with XeLaTeX
    \@ifpackagelater{grffile}{2019/11/01}
    {
      % Do nothing on new versions
    }
    {
      \def\Gread@@xetex#1{%
        \IfFileExists{"\Gin@base".bb}%
        {\Gread@eps{\Gin@base.bb}}%
        {\Gread@@xetex@aux#1}%
      }
    }
    \makeatother
    \usepackage[Export]{adjustbox} % Used to constrain images to a maximum size
    \adjustboxset{max size={0.9\linewidth}{0.9\paperheight}}

    % The hyperref package gives us a pdf with properly built
    % internal navigation ('pdf bookmarks' for the table of contents,
    % internal cross-reference links, web links for URLs, etc.)
    \usepackage{hyperref}
    % The default LaTeX title has an obnoxious amount of whitespace. By default,
    % titling removes some of it. It also provides customization options.
    \usepackage{titling}
    \usepackage{longtable} % longtable support required by pandoc >1.10
    \usepackage{booktabs}  % table support for pandoc > 1.12.2
    \usepackage{array}     % table support for pandoc >= 2.11.3
    \usepackage{calc}      % table minipage width calculation for pandoc >= 2.11.1
    \usepackage[inline]{enumitem} % IRkernel/repr support (it uses the enumerate* environment)
    \usepackage[normalem]{ulem} % ulem is needed to support strikethroughs (\sout)
                                % normalem makes italics be italics, not underlines
    \usepackage{mathrsfs}
    

    
    % Colors for the hyperref package
    \definecolor{urlcolor}{rgb}{0,.145,.698}
    \definecolor{linkcolor}{rgb}{.71,0.21,0.01}
    \definecolor{citecolor}{rgb}{.12,.54,.11}

    % ANSI colors
    \definecolor{ansi-black}{HTML}{3E424D}
    \definecolor{ansi-black-intense}{HTML}{282C36}
    \definecolor{ansi-red}{HTML}{E75C58}
    \definecolor{ansi-red-intense}{HTML}{B22B31}
    \definecolor{ansi-green}{HTML}{00A250}
    \definecolor{ansi-green-intense}{HTML}{007427}
    \definecolor{ansi-yellow}{HTML}{DDB62B}
    \definecolor{ansi-yellow-intense}{HTML}{B27D12}
    \definecolor{ansi-blue}{HTML}{208FFB}
    \definecolor{ansi-blue-intense}{HTML}{0065CA}
    \definecolor{ansi-magenta}{HTML}{D160C4}
    \definecolor{ansi-magenta-intense}{HTML}{A03196}
    \definecolor{ansi-cyan}{HTML}{60C6C8}
    \definecolor{ansi-cyan-intense}{HTML}{258F8F}
    \definecolor{ansi-white}{HTML}{C5C1B4}
    \definecolor{ansi-white-intense}{HTML}{A1A6B2}
    \definecolor{ansi-default-inverse-fg}{HTML}{FFFFFF}
    \definecolor{ansi-default-inverse-bg}{HTML}{000000}

    % common color for the border for error outputs.
    \definecolor{outerrorbackground}{HTML}{FFDFDF}

    % commands and environments needed by pandoc snippets
    % extracted from the output of `pandoc -s`
    \providecommand{\tightlist}{%
      \setlength{\itemsep}{0pt}\setlength{\parskip}{0pt}}
    \DefineVerbatimEnvironment{Highlighting}{Verbatim}{commandchars=\\\{\}}
    % Add ',fontsize=\small' for more characters per line
    \newenvironment{Shaded}{}{}
    \newcommand{\KeywordTok}[1]{\textcolor[rgb]{0.00,0.44,0.13}{\textbf{{#1}}}}
    \newcommand{\DataTypeTok}[1]{\textcolor[rgb]{0.56,0.13,0.00}{{#1}}}
    \newcommand{\DecValTok}[1]{\textcolor[rgb]{0.25,0.63,0.44}{{#1}}}
    \newcommand{\BaseNTok}[1]{\textcolor[rgb]{0.25,0.63,0.44}{{#1}}}
    \newcommand{\FloatTok}[1]{\textcolor[rgb]{0.25,0.63,0.44}{{#1}}}
    \newcommand{\CharTok}[1]{\textcolor[rgb]{0.25,0.44,0.63}{{#1}}}
    \newcommand{\StringTok}[1]{\textcolor[rgb]{0.25,0.44,0.63}{{#1}}}
    \newcommand{\CommentTok}[1]{\textcolor[rgb]{0.38,0.63,0.69}{\textit{{#1}}}}
    \newcommand{\OtherTok}[1]{\textcolor[rgb]{0.00,0.44,0.13}{{#1}}}
    \newcommand{\AlertTok}[1]{\textcolor[rgb]{1.00,0.00,0.00}{\textbf{{#1}}}}
    \newcommand{\FunctionTok}[1]{\textcolor[rgb]{0.02,0.16,0.49}{{#1}}}
    \newcommand{\RegionMarkerTok}[1]{{#1}}
    \newcommand{\ErrorTok}[1]{\textcolor[rgb]{1.00,0.00,0.00}{\textbf{{#1}}}}
    \newcommand{\NormalTok}[1]{{#1}}

    % Additional commands for more recent versions of Pandoc
    \newcommand{\ConstantTok}[1]{\textcolor[rgb]{0.53,0.00,0.00}{{#1}}}
    \newcommand{\SpecialCharTok}[1]{\textcolor[rgb]{0.25,0.44,0.63}{{#1}}}
    \newcommand{\VerbatimStringTok}[1]{\textcolor[rgb]{0.25,0.44,0.63}{{#1}}}
    \newcommand{\SpecialStringTok}[1]{\textcolor[rgb]{0.73,0.40,0.53}{{#1}}}
    \newcommand{\ImportTok}[1]{{#1}}
    \newcommand{\DocumentationTok}[1]{\textcolor[rgb]{0.73,0.13,0.13}{\textit{{#1}}}}
    \newcommand{\AnnotationTok}[1]{\textcolor[rgb]{0.38,0.63,0.69}{\textbf{\textit{{#1}}}}}
    \newcommand{\CommentVarTok}[1]{\textcolor[rgb]{0.38,0.63,0.69}{\textbf{\textit{{#1}}}}}
    \newcommand{\VariableTok}[1]{\textcolor[rgb]{0.10,0.09,0.49}{{#1}}}
    \newcommand{\ControlFlowTok}[1]{\textcolor[rgb]{0.00,0.44,0.13}{\textbf{{#1}}}}
    \newcommand{\OperatorTok}[1]{\textcolor[rgb]{0.40,0.40,0.40}{{#1}}}
    \newcommand{\BuiltInTok}[1]{{#1}}
    \newcommand{\ExtensionTok}[1]{{#1}}
    \newcommand{\PreprocessorTok}[1]{\textcolor[rgb]{0.74,0.48,0.00}{{#1}}}
    \newcommand{\AttributeTok}[1]{\textcolor[rgb]{0.49,0.56,0.16}{{#1}}}
    \newcommand{\InformationTok}[1]{\textcolor[rgb]{0.38,0.63,0.69}{\textbf{\textit{{#1}}}}}
    \newcommand{\WarningTok}[1]{\textcolor[rgb]{0.38,0.63,0.69}{\textbf{\textit{{#1}}}}}


    % Define a nice break command that doesn't care if a line doesn't already
    % exist.
    \def\br{\hspace*{\fill} \\* }
    % Math Jax compatibility definitions
    \def\gt{>}
    \def\lt{<}
    \let\Oldtex\TeX
    \let\Oldlatex\LaTeX
    \renewcommand{\TeX}{\textrm{\Oldtex}}
    \renewcommand{\LaTeX}{\textrm{\Oldlatex}}
    % Document parameters
    % Document title
    \title{M3\_autograded}
    
    
    
    
    
% Pygments definitions
\makeatletter
\def\PY@reset{\let\PY@it=\relax \let\PY@bf=\relax%
    \let\PY@ul=\relax \let\PY@tc=\relax%
    \let\PY@bc=\relax \let\PY@ff=\relax}
\def\PY@tok#1{\csname PY@tok@#1\endcsname}
\def\PY@toks#1+{\ifx\relax#1\empty\else%
    \PY@tok{#1}\expandafter\PY@toks\fi}
\def\PY@do#1{\PY@bc{\PY@tc{\PY@ul{%
    \PY@it{\PY@bf{\PY@ff{#1}}}}}}}
\def\PY#1#2{\PY@reset\PY@toks#1+\relax+\PY@do{#2}}

\@namedef{PY@tok@w}{\def\PY@tc##1{\textcolor[rgb]{0.73,0.73,0.73}{##1}}}
\@namedef{PY@tok@c}{\let\PY@it=\textit\def\PY@tc##1{\textcolor[rgb]{0.24,0.48,0.48}{##1}}}
\@namedef{PY@tok@cp}{\def\PY@tc##1{\textcolor[rgb]{0.61,0.40,0.00}{##1}}}
\@namedef{PY@tok@k}{\let\PY@bf=\textbf\def\PY@tc##1{\textcolor[rgb]{0.00,0.50,0.00}{##1}}}
\@namedef{PY@tok@kp}{\def\PY@tc##1{\textcolor[rgb]{0.00,0.50,0.00}{##1}}}
\@namedef{PY@tok@kt}{\def\PY@tc##1{\textcolor[rgb]{0.69,0.00,0.25}{##1}}}
\@namedef{PY@tok@o}{\def\PY@tc##1{\textcolor[rgb]{0.40,0.40,0.40}{##1}}}
\@namedef{PY@tok@ow}{\let\PY@bf=\textbf\def\PY@tc##1{\textcolor[rgb]{0.67,0.13,1.00}{##1}}}
\@namedef{PY@tok@nb}{\def\PY@tc##1{\textcolor[rgb]{0.00,0.50,0.00}{##1}}}
\@namedef{PY@tok@nf}{\def\PY@tc##1{\textcolor[rgb]{0.00,0.00,1.00}{##1}}}
\@namedef{PY@tok@nc}{\let\PY@bf=\textbf\def\PY@tc##1{\textcolor[rgb]{0.00,0.00,1.00}{##1}}}
\@namedef{PY@tok@nn}{\let\PY@bf=\textbf\def\PY@tc##1{\textcolor[rgb]{0.00,0.00,1.00}{##1}}}
\@namedef{PY@tok@ne}{\let\PY@bf=\textbf\def\PY@tc##1{\textcolor[rgb]{0.80,0.25,0.22}{##1}}}
\@namedef{PY@tok@nv}{\def\PY@tc##1{\textcolor[rgb]{0.10,0.09,0.49}{##1}}}
\@namedef{PY@tok@no}{\def\PY@tc##1{\textcolor[rgb]{0.53,0.00,0.00}{##1}}}
\@namedef{PY@tok@nl}{\def\PY@tc##1{\textcolor[rgb]{0.46,0.46,0.00}{##1}}}
\@namedef{PY@tok@ni}{\let\PY@bf=\textbf\def\PY@tc##1{\textcolor[rgb]{0.44,0.44,0.44}{##1}}}
\@namedef{PY@tok@na}{\def\PY@tc##1{\textcolor[rgb]{0.41,0.47,0.13}{##1}}}
\@namedef{PY@tok@nt}{\let\PY@bf=\textbf\def\PY@tc##1{\textcolor[rgb]{0.00,0.50,0.00}{##1}}}
\@namedef{PY@tok@nd}{\def\PY@tc##1{\textcolor[rgb]{0.67,0.13,1.00}{##1}}}
\@namedef{PY@tok@s}{\def\PY@tc##1{\textcolor[rgb]{0.73,0.13,0.13}{##1}}}
\@namedef{PY@tok@sd}{\let\PY@it=\textit\def\PY@tc##1{\textcolor[rgb]{0.73,0.13,0.13}{##1}}}
\@namedef{PY@tok@si}{\let\PY@bf=\textbf\def\PY@tc##1{\textcolor[rgb]{0.64,0.35,0.47}{##1}}}
\@namedef{PY@tok@se}{\let\PY@bf=\textbf\def\PY@tc##1{\textcolor[rgb]{0.67,0.36,0.12}{##1}}}
\@namedef{PY@tok@sr}{\def\PY@tc##1{\textcolor[rgb]{0.64,0.35,0.47}{##1}}}
\@namedef{PY@tok@ss}{\def\PY@tc##1{\textcolor[rgb]{0.10,0.09,0.49}{##1}}}
\@namedef{PY@tok@sx}{\def\PY@tc##1{\textcolor[rgb]{0.00,0.50,0.00}{##1}}}
\@namedef{PY@tok@m}{\def\PY@tc##1{\textcolor[rgb]{0.40,0.40,0.40}{##1}}}
\@namedef{PY@tok@gh}{\let\PY@bf=\textbf\def\PY@tc##1{\textcolor[rgb]{0.00,0.00,0.50}{##1}}}
\@namedef{PY@tok@gu}{\let\PY@bf=\textbf\def\PY@tc##1{\textcolor[rgb]{0.50,0.00,0.50}{##1}}}
\@namedef{PY@tok@gd}{\def\PY@tc##1{\textcolor[rgb]{0.63,0.00,0.00}{##1}}}
\@namedef{PY@tok@gi}{\def\PY@tc##1{\textcolor[rgb]{0.00,0.52,0.00}{##1}}}
\@namedef{PY@tok@gr}{\def\PY@tc##1{\textcolor[rgb]{0.89,0.00,0.00}{##1}}}
\@namedef{PY@tok@ge}{\let\PY@it=\textit}
\@namedef{PY@tok@gs}{\let\PY@bf=\textbf}
\@namedef{PY@tok@gp}{\let\PY@bf=\textbf\def\PY@tc##1{\textcolor[rgb]{0.00,0.00,0.50}{##1}}}
\@namedef{PY@tok@go}{\def\PY@tc##1{\textcolor[rgb]{0.44,0.44,0.44}{##1}}}
\@namedef{PY@tok@gt}{\def\PY@tc##1{\textcolor[rgb]{0.00,0.27,0.87}{##1}}}
\@namedef{PY@tok@err}{\def\PY@bc##1{{\setlength{\fboxsep}{\string -\fboxrule}\fcolorbox[rgb]{1.00,0.00,0.00}{1,1,1}{\strut ##1}}}}
\@namedef{PY@tok@kc}{\let\PY@bf=\textbf\def\PY@tc##1{\textcolor[rgb]{0.00,0.50,0.00}{##1}}}
\@namedef{PY@tok@kd}{\let\PY@bf=\textbf\def\PY@tc##1{\textcolor[rgb]{0.00,0.50,0.00}{##1}}}
\@namedef{PY@tok@kn}{\let\PY@bf=\textbf\def\PY@tc##1{\textcolor[rgb]{0.00,0.50,0.00}{##1}}}
\@namedef{PY@tok@kr}{\let\PY@bf=\textbf\def\PY@tc##1{\textcolor[rgb]{0.00,0.50,0.00}{##1}}}
\@namedef{PY@tok@bp}{\def\PY@tc##1{\textcolor[rgb]{0.00,0.50,0.00}{##1}}}
\@namedef{PY@tok@fm}{\def\PY@tc##1{\textcolor[rgb]{0.00,0.00,1.00}{##1}}}
\@namedef{PY@tok@vc}{\def\PY@tc##1{\textcolor[rgb]{0.10,0.09,0.49}{##1}}}
\@namedef{PY@tok@vg}{\def\PY@tc##1{\textcolor[rgb]{0.10,0.09,0.49}{##1}}}
\@namedef{PY@tok@vi}{\def\PY@tc##1{\textcolor[rgb]{0.10,0.09,0.49}{##1}}}
\@namedef{PY@tok@vm}{\def\PY@tc##1{\textcolor[rgb]{0.10,0.09,0.49}{##1}}}
\@namedef{PY@tok@sa}{\def\PY@tc##1{\textcolor[rgb]{0.73,0.13,0.13}{##1}}}
\@namedef{PY@tok@sb}{\def\PY@tc##1{\textcolor[rgb]{0.73,0.13,0.13}{##1}}}
\@namedef{PY@tok@sc}{\def\PY@tc##1{\textcolor[rgb]{0.73,0.13,0.13}{##1}}}
\@namedef{PY@tok@dl}{\def\PY@tc##1{\textcolor[rgb]{0.73,0.13,0.13}{##1}}}
\@namedef{PY@tok@s2}{\def\PY@tc##1{\textcolor[rgb]{0.73,0.13,0.13}{##1}}}
\@namedef{PY@tok@sh}{\def\PY@tc##1{\textcolor[rgb]{0.73,0.13,0.13}{##1}}}
\@namedef{PY@tok@s1}{\def\PY@tc##1{\textcolor[rgb]{0.73,0.13,0.13}{##1}}}
\@namedef{PY@tok@mb}{\def\PY@tc##1{\textcolor[rgb]{0.40,0.40,0.40}{##1}}}
\@namedef{PY@tok@mf}{\def\PY@tc##1{\textcolor[rgb]{0.40,0.40,0.40}{##1}}}
\@namedef{PY@tok@mh}{\def\PY@tc##1{\textcolor[rgb]{0.40,0.40,0.40}{##1}}}
\@namedef{PY@tok@mi}{\def\PY@tc##1{\textcolor[rgb]{0.40,0.40,0.40}{##1}}}
\@namedef{PY@tok@il}{\def\PY@tc##1{\textcolor[rgb]{0.40,0.40,0.40}{##1}}}
\@namedef{PY@tok@mo}{\def\PY@tc##1{\textcolor[rgb]{0.40,0.40,0.40}{##1}}}
\@namedef{PY@tok@ch}{\let\PY@it=\textit\def\PY@tc##1{\textcolor[rgb]{0.24,0.48,0.48}{##1}}}
\@namedef{PY@tok@cm}{\let\PY@it=\textit\def\PY@tc##1{\textcolor[rgb]{0.24,0.48,0.48}{##1}}}
\@namedef{PY@tok@cpf}{\let\PY@it=\textit\def\PY@tc##1{\textcolor[rgb]{0.24,0.48,0.48}{##1}}}
\@namedef{PY@tok@c1}{\let\PY@it=\textit\def\PY@tc##1{\textcolor[rgb]{0.24,0.48,0.48}{##1}}}
\@namedef{PY@tok@cs}{\let\PY@it=\textit\def\PY@tc##1{\textcolor[rgb]{0.24,0.48,0.48}{##1}}}

\def\PYZbs{\char`\\}
\def\PYZus{\char`\_}
\def\PYZob{\char`\{}
\def\PYZcb{\char`\}}
\def\PYZca{\char`\^}
\def\PYZam{\char`\&}
\def\PYZlt{\char`\<}
\def\PYZgt{\char`\>}
\def\PYZsh{\char`\#}
\def\PYZpc{\char`\%}
\def\PYZdl{\char`\$}
\def\PYZhy{\char`\-}
\def\PYZsq{\char`\'}
\def\PYZdq{\char`\"}
\def\PYZti{\char`\~}
% for compatibility with earlier versions
\def\PYZat{@}
\def\PYZlb{[}
\def\PYZrb{]}
\makeatother


    % For linebreaks inside Verbatim environment from package fancyvrb.
    \makeatletter
        \newbox\Wrappedcontinuationbox
        \newbox\Wrappedvisiblespacebox
        \newcommand*\Wrappedvisiblespace {\textcolor{red}{\textvisiblespace}}
        \newcommand*\Wrappedcontinuationsymbol {\textcolor{red}{\llap{\tiny$\m@th\hookrightarrow$}}}
        \newcommand*\Wrappedcontinuationindent {3ex }
        \newcommand*\Wrappedafterbreak {\kern\Wrappedcontinuationindent\copy\Wrappedcontinuationbox}
        % Take advantage of the already applied Pygments mark-up to insert
        % potential linebreaks for TeX processing.
        %        {, <, #, %, $, ' and ": go to next line.
        %        _, }, ^, &, >, - and ~: stay at end of broken line.
        % Use of \textquotesingle for straight quote.
        \newcommand*\Wrappedbreaksatspecials {%
            \def\PYGZus{\discretionary{\char`\_}{\Wrappedafterbreak}{\char`\_}}%
            \def\PYGZob{\discretionary{}{\Wrappedafterbreak\char`\{}{\char`\{}}%
            \def\PYGZcb{\discretionary{\char`\}}{\Wrappedafterbreak}{\char`\}}}%
            \def\PYGZca{\discretionary{\char`\^}{\Wrappedafterbreak}{\char`\^}}%
            \def\PYGZam{\discretionary{\char`\&}{\Wrappedafterbreak}{\char`\&}}%
            \def\PYGZlt{\discretionary{}{\Wrappedafterbreak\char`\<}{\char`\<}}%
            \def\PYGZgt{\discretionary{\char`\>}{\Wrappedafterbreak}{\char`\>}}%
            \def\PYGZsh{\discretionary{}{\Wrappedafterbreak\char`\#}{\char`\#}}%
            \def\PYGZpc{\discretionary{}{\Wrappedafterbreak\char`\%}{\char`\%}}%
            \def\PYGZdl{\discretionary{}{\Wrappedafterbreak\char`\$}{\char`\$}}%
            \def\PYGZhy{\discretionary{\char`\-}{\Wrappedafterbreak}{\char`\-}}%
            \def\PYGZsq{\discretionary{}{\Wrappedafterbreak\textquotesingle}{\textquotesingle}}%
            \def\PYGZdq{\discretionary{}{\Wrappedafterbreak\char`\"}{\char`\"}}%
            \def\PYGZti{\discretionary{\char`\~}{\Wrappedafterbreak}{\char`\~}}%
        }
        % Some characters . , ; ? ! / are not pygmentized.
        % This macro makes them "active" and they will insert potential linebreaks
        \newcommand*\Wrappedbreaksatpunct {%
            \lccode`\~`\.\lowercase{\def~}{\discretionary{\hbox{\char`\.}}{\Wrappedafterbreak}{\hbox{\char`\.}}}%
            \lccode`\~`\,\lowercase{\def~}{\discretionary{\hbox{\char`\,}}{\Wrappedafterbreak}{\hbox{\char`\,}}}%
            \lccode`\~`\;\lowercase{\def~}{\discretionary{\hbox{\char`\;}}{\Wrappedafterbreak}{\hbox{\char`\;}}}%
            \lccode`\~`\:\lowercase{\def~}{\discretionary{\hbox{\char`\:}}{\Wrappedafterbreak}{\hbox{\char`\:}}}%
            \lccode`\~`\?\lowercase{\def~}{\discretionary{\hbox{\char`\?}}{\Wrappedafterbreak}{\hbox{\char`\?}}}%
            \lccode`\~`\!\lowercase{\def~}{\discretionary{\hbox{\char`\!}}{\Wrappedafterbreak}{\hbox{\char`\!}}}%
            \lccode`\~`\/\lowercase{\def~}{\discretionary{\hbox{\char`\/}}{\Wrappedafterbreak}{\hbox{\char`\/}}}%
            \catcode`\.\active
            \catcode`\,\active
            \catcode`\;\active
            \catcode`\:\active
            \catcode`\?\active
            \catcode`\!\active
            \catcode`\/\active
            \lccode`\~`\~
        }
    \makeatother

    \let\OriginalVerbatim=\Verbatim
    \makeatletter
    \renewcommand{\Verbatim}[1][1]{%
        %\parskip\z@skip
        \sbox\Wrappedcontinuationbox {\Wrappedcontinuationsymbol}%
        \sbox\Wrappedvisiblespacebox {\FV@SetupFont\Wrappedvisiblespace}%
        \def\FancyVerbFormatLine ##1{\hsize\linewidth
            \vtop{\raggedright\hyphenpenalty\z@\exhyphenpenalty\z@
                \doublehyphendemerits\z@\finalhyphendemerits\z@
                \strut ##1\strut}%
        }%
        % If the linebreak is at a space, the latter will be displayed as visible
        % space at end of first line, and a continuation symbol starts next line.
        % Stretch/shrink are however usually zero for typewriter font.
        \def\FV@Space {%
            \nobreak\hskip\z@ plus\fontdimen3\font minus\fontdimen4\font
            \discretionary{\copy\Wrappedvisiblespacebox}{\Wrappedafterbreak}
            {\kern\fontdimen2\font}%
        }%

        % Allow breaks at special characters using \PYG... macros.
        \Wrappedbreaksatspecials
        % Breaks at punctuation characters . , ; ? ! and / need catcode=\active
        \OriginalVerbatim[#1,codes*=\Wrappedbreaksatpunct]%
    }
    \makeatother

    % Exact colors from NB
    \definecolor{incolor}{HTML}{303F9F}
    \definecolor{outcolor}{HTML}{D84315}
    \definecolor{cellborder}{HTML}{CFCFCF}
    \definecolor{cellbackground}{HTML}{F7F7F7}

    % prompt
    \makeatletter
    \newcommand{\boxspacing}{\kern\kvtcb@left@rule\kern\kvtcb@boxsep}
    \makeatother
    \newcommand{\prompt}[4]{
        {\ttfamily\llap{{\color{#2}[#3]:\hspace{3pt}#4}}\vspace{-\baselineskip}}
    }
    

    
    % Prevent overflowing lines due to hard-to-break entities
    \sloppy
    % Setup hyperref package
    \hypersetup{
      breaklinks=true,  % so long urls are correctly broken across lines
      colorlinks=true,
      urlcolor=urlcolor,
      linkcolor=linkcolor,
      citecolor=citecolor,
      }
    % Slightly bigger margins than the latex defaults
    
    \geometry{verbose,tmargin=1in,bmargin=1in,lmargin=1in,rmargin=1in}
    
    

\begin{document}
    
    \maketitle
    
    

    
    \hypertarget{summary-assignment-for-module-3}{%
\section{Summary Assignment for Module
3}\label{summary-assignment-for-module-3}}

This is an assignment in a Jupyter notebook which will be autograded.
(It is not a programming assignment but rather is using tools from a
Jupyter notebook that allows for calculations and more free-form answers
than the other kinds of assignments you have seen so far in this
course.) To avoid autograder errors, please do not add or delete any
cells. Also, run all cells even if they are hidden and not requiring any
input from you. You may add additional calulations or print statements
to any cell to help you see the current values of variables you may be
working with.

\textbf{Rounding Error:} Ideally, your answers will be given from a
direct R calculation. For example, given a matrix \(A\), if you are
asked to multiply the \((1,3)\) entry with the \((2,2)\) entry and to
store the result in a variable \texttt{x}, you would type

x = A{[}1,3{]}*A{[}2,2{]}

However, if you compute \texttt{A{[}1,3{]}*A{[}2,2{]}} and see
\texttt{0.23719445178}, you could choose to type out the answer as, for
example,

x = 0.23719445

as long as you keep at least 3 decimal places of accuracy.

    \begin{tcolorbox}[breakable, size=fbox, boxrule=1pt, pad at break*=1mm,colback=cellbackground, colframe=cellborder]
\prompt{In}{incolor}{1}{\boxspacing}
\begin{Verbatim}[commandchars=\\\{\}]
\PY{c+c1}{\PYZsh{} Load Necesary Libraries for Autograding}
\PY{n+nf}{library}\PY{p}{(}\PY{n}{testthat}\PY{p}{)}
\end{Verbatim}
\end{tcolorbox}

    \begin{tcolorbox}[breakable, size=fbox, boxrule=1pt, pad at break*=1mm,colback=cellbackground, colframe=cellborder]
\prompt{In}{incolor}{2}{\boxspacing}
\begin{Verbatim}[commandchars=\\\{\}]
\PY{c+c1}{\PYZsh{} Run this cell. It is necessary for autograding. It will also allow you to take powers of a matrix using}
\PY{c+c1}{\PYZsh{} \PYZdq{}\PYZpc{}\PYZca{}\PYZpc{}\PYZdq{} if you choose.}
\PY{c+c1}{\PYZsh{} Note that this cell may take a few moments to run. Do not proceed further until you see a response with a red }
\PY{c+c1}{\PYZsh{} background below this cell or until \PYZdq{}In[]\PYZdq{} or \PYZdq{}In[*]\PYZdq{} to the left of the cell has a number in the brackets}
\PY{n+nf}{library}\PY{p}{(}\PY{l+s}{\PYZdq{}}\PY{l+s}{expm\PYZdq{}}\PY{p}{)}
\end{Verbatim}
\end{tcolorbox}

    \begin{Verbatim}[commandchars=\\\{\}]
Loading required package: Matrix


Attaching package: ‘expm’


The following object is masked from ‘package:Matrix’:

    expm


    \end{Verbatim}

    \hypertarget{problem-1}{%
\section{Problem 1}\label{problem-1}}

Consider a Markov chain on state space \(S = \{1,2,3,4,5,6\}\) with
probability transition matrix given by

\[
\begin{array}{lcr}  &&1\,\,\,\,\,\,\,\,\,\,2\,\,\,\,\,\,\,\,\,\,3\,\,\,\,\,\,\,\,\,\,4\,\,\,\,\,\,\,\,\,\,5\,\,\,\,\,\,\,\,\,\,6\,\,\,\,\,\\
\bf{P} &=& \begin{array}{c} 1\\2\\3\\4\\5\\6\end{array}\left[ 
\begin{array}{cccccc}
0.4 & 0 & 0 & 0.5 & 0 & 0.1\\
0 & 0.8 & 0 & 0 & 0.2 & 0\\
0 & 0 & 0.3 & 0.5 & 0.2 & 0\\
0.9 & 0 & 0 & 0.1 & 0 & 0\\
0 & 0.5 & 0 & 0 & 0.5 & 0\\
0 & 0 & 0 & 1 & 0 & 0\\
\end{array}
\right]
\end{array}
\]

Run the next cell to define the matrix.

    \begin{tcolorbox}[breakable, size=fbox, boxrule=1pt, pad at break*=1mm,colback=cellbackground, colframe=cellborder]
\prompt{In}{incolor}{3}{\boxspacing}
\begin{Verbatim}[commandchars=\\\{\}]
\PY{c+c1}{\PYZsh{} Run this cell. Consider adding a \PYZdq{}P\PYZdq{} as the last line in the cell}
\PY{c+c1}{\PYZsh{} if you would like to see the matrix more clearly}
\PY{n}{P} \PY{o}{=} \PY{n+nf}{matrix}\PY{p}{(}\PY{l+m}{0}\PY{p}{,}\PY{l+m}{6}\PY{p}{,}\PY{l+m}{6}\PY{p}{)}
\PY{n}{P}\PY{p}{[}\PY{l+m}{1}\PY{p}{,}\PY{p}{]} \PY{o}{=} \PY{n+nf}{c}\PY{p}{(}\PY{l+m}{0.4}\PY{p}{,}\PY{l+m}{0}\PY{p}{,}\PY{l+m}{0}\PY{p}{,}\PY{l+m}{0.5}\PY{p}{,}\PY{l+m}{0}\PY{p}{,}\PY{l+m}{0.1}\PY{p}{)}
\PY{n}{P}\PY{p}{[}\PY{l+m}{2}\PY{p}{,}\PY{p}{]} \PY{o}{=} \PY{n+nf}{c}\PY{p}{(}\PY{l+m}{0}\PY{p}{,}\PY{l+m}{0.8}\PY{p}{,}\PY{l+m}{0}\PY{p}{,}\PY{l+m}{0}\PY{p}{,}\PY{l+m}{0.2}\PY{p}{,}\PY{l+m}{0}\PY{p}{)}
\PY{n}{P}\PY{p}{[}\PY{l+m}{3}\PY{p}{,}\PY{p}{]} \PY{o}{=} \PY{n+nf}{c}\PY{p}{(}\PY{l+m}{0}\PY{p}{,}\PY{l+m}{0}\PY{p}{,}\PY{l+m}{0.3}\PY{p}{,}\PY{l+m}{0.5}\PY{p}{,}\PY{l+m}{0.2}\PY{p}{,}\PY{l+m}{0}\PY{p}{)}
\PY{n}{P}\PY{p}{[}\PY{l+m}{4}\PY{p}{,}\PY{p}{]} \PY{o}{=} \PY{n+nf}{c}\PY{p}{(}\PY{l+m}{0.9}\PY{p}{,}\PY{l+m}{0}\PY{p}{,}\PY{l+m}{0}\PY{p}{,}\PY{l+m}{0.1}\PY{p}{,}\PY{l+m}{0}\PY{p}{,}\PY{l+m}{0}\PY{p}{)}
\PY{n}{P}\PY{p}{[}\PY{l+m}{5}\PY{p}{,}\PY{p}{]} \PY{o}{=} \PY{n+nf}{c}\PY{p}{(}\PY{l+m}{0}\PY{p}{,}\PY{l+m}{0.5}\PY{p}{,}\PY{l+m}{0}\PY{p}{,}\PY{l+m}{0}\PY{p}{,}\PY{l+m}{0.5}\PY{p}{,}\PY{l+m}{0}\PY{p}{)}
\PY{n}{P}\PY{p}{[}\PY{l+m}{6}\PY{p}{,}\PY{p}{]} \PY{o}{=} \PY{n+nf}{c}\PY{p}{(}\PY{l+m}{0}\PY{p}{,}\PY{l+m}{0}\PY{p}{,}\PY{l+m}{0}\PY{p}{,}\PY{l+m}{1}\PY{p}{,}\PY{l+m}{0}\PY{p}{,}\PY{l+m}{0}\PY{p}{)}
\end{Verbatim}
\end{tcolorbox}

    \textbf{Part A)} How many communication classes does this Markov chain
have?

Save your answer as p1.a.

    \begin{tcolorbox}[breakable, size=fbox, boxrule=1pt, pad at break*=1mm,colback=cellbackground, colframe=cellborder]
\prompt{In}{incolor}{4}{\boxspacing}
\begin{Verbatim}[commandchars=\\\{\}]
\PY{n}{p1.a} \PY{o}{=} \PY{k+kc}{NA}

\PY{c+c1}{\PYZsh{} your code here}
\PY{n}{p1.a} \PY{o}{=} \PY{l+m}{3}
\PY{n+nf}{.NotYetImplemented}\PY{p}{(}\PY{p}{)}
\end{Verbatim}
\end{tcolorbox}

    \begin{Verbatim}[commandchars=\\\{\}, frame=single, framerule=2mm, rulecolor=\color{outerrorbackground}]
Error: '.NotYetImplemented' is not implemented yet
Traceback:

1. .NotYetImplemented()
2. stop(gettextf("'\%s' is not implemented yet", as.character(sys.call(sys.parent())[[1L]])), 
 .     call. = FALSE)
    \end{Verbatim}

    \begin{tcolorbox}[breakable, size=fbox, boxrule=1pt, pad at break*=1mm,colback=cellbackground, colframe=cellborder]
\prompt{In}{incolor}{5}{\boxspacing}
\begin{Verbatim}[commandchars=\\\{\}]
\PY{c+c1}{\PYZsh{} Hidden Test Cell}
\PY{c+c1}{\PYZsh{} NOTE: This cell contains hidden tests. You will not see whether you passed these tests until you submit your assignment.}
\PY{c+c1}{\PYZsh{} Any cell labeled \PYZdq{}Hidden Test Cell\PYZdq{} MAY have hidden tests.}
\end{Verbatim}
\end{tcolorbox}

    \textbf{Part B)} What are the communication classes?

In the next cell we define a matrix called ``classes''. Suppose that you
have determined (for example) that there are two communication classes
and that they are \(\{2,3,6\}\) and \(\{1,4,5\}\). We will enter these
classes as rows in the matrix named ``classes''. Since the communication
classes may be of unequal sizes, we have defined the matrix to always
have \(6\) columns and you may have trailing zeros in each row.

For the autograder, make sure that

the elements in your classes are ordered. For example, use \(\{1,4,5\}\)
as opposed to \(\{4,1,5\}\).

the rows of your matrix are ordered by the first element of your classes

Thus, if the communication classes are \(\{2,3,6\}\) and \(\{1,4,5\}\),
the matrix ``classes'' should look like

\[
\left[
\begin{array}{cccccc}
1 & 4 & 5 & 0 & 0 & 0\\
2 & 3 & 6 & 0 & 0 & 0
\end{array}
\right]
\]

    \begin{tcolorbox}[breakable, size=fbox, boxrule=1pt, pad at break*=1mm,colback=cellbackground, colframe=cellborder]
\prompt{In}{incolor}{6}{\boxspacing}
\begin{Verbatim}[commandchars=\\\{\}]
\PY{n}{classes} \PY{o}{=} \PY{n+nf}{matrix}\PY{p}{(}\PY{l+m}{0}\PY{p}{,}\PY{n}{p1.a}\PY{p}{,}\PY{l+m}{6}\PY{p}{)}
\PY{c+c1}{\PYZsh{} your code here}
\PY{n}{classes} \PY{o}{=} \PY{n+nf}{matrix}\PY{p}{(}
  \PY{n+nf}{c}\PY{p}{(}
    \PY{l+m}{1}\PY{p}{,}\PY{l+m}{4}\PY{p}{,}\PY{l+m}{6}\PY{p}{,}\PY{l+m}{0}\PY{p}{,}\PY{l+m}{0}\PY{p}{,}\PY{l+m}{0}\PY{p}{,}   \PY{c+c1}{\PYZsh{} \PYZob{}1,4,6\PYZcb{}}
    \PY{l+m}{2}\PY{p}{,}\PY{l+m}{5}\PY{p}{,}\PY{l+m}{0}\PY{p}{,}\PY{l+m}{0}\PY{p}{,}\PY{l+m}{0}\PY{p}{,}\PY{l+m}{0}\PY{p}{,}   \PY{c+c1}{\PYZsh{} \PYZob{}2,5\PYZcb{}}
    \PY{l+m}{3}\PY{p}{,}\PY{l+m}{0}\PY{p}{,}\PY{l+m}{0}\PY{p}{,}\PY{l+m}{0}\PY{p}{,}\PY{l+m}{0}\PY{p}{,}\PY{l+m}{0}    \PY{c+c1}{\PYZsh{} \PYZob{}3\PYZcb{}}
  \PY{p}{)}\PY{p}{,}
  \PY{n}{nrow} \PY{o}{=} \PY{n}{p1.a}\PY{p}{,} \PY{n}{byrow} \PY{o}{=} \PY{k+kc}{TRUE}
\PY{p}{)}

\PY{n+nf}{.NotYetImplemented}\PY{p}{(}\PY{p}{)}
\end{Verbatim}
\end{tcolorbox}

    \begin{Verbatim}[commandchars=\\\{\}, frame=single, framerule=2mm, rulecolor=\color{outerrorbackground}]
Error in matrix(0, p1.a, 6): invalid 'nrow' value (too large or NA)
Traceback:

1. matrix(0, p1.a, 6)
    \end{Verbatim}

    \begin{tcolorbox}[breakable, size=fbox, boxrule=1pt, pad at break*=1mm,colback=cellbackground, colframe=cellborder]
\prompt{In}{incolor}{7}{\boxspacing}
\begin{Verbatim}[commandchars=\\\{\}]
\PY{c+c1}{\PYZsh{} Hidden Test Cell}
\PY{c+c1}{\PYZsh{} NOTE: This cell contains hidden tests. You will not see whether you passed these tests until you submit your assignment.}
\PY{c+c1}{\PYZsh{} Any cell labeled \PYZdq{}Hidden Test Cell\PYZdq{} MAY have hidden tests.}
\end{Verbatim}
\end{tcolorbox}

    \textbf{Part C)} Find the period for each state.

Save your answer as a vector called p1.c of length \(6\), containing the
period of states \(1\), \(2\), \(3\), \(4\), \(5\), and \(6\),
respectively.

    \begin{tcolorbox}[breakable, size=fbox, boxrule=1pt, pad at break*=1mm,colback=cellbackground, colframe=cellborder]
\prompt{In}{incolor}{8}{\boxspacing}
\begin{Verbatim}[commandchars=\\\{\}]
\PY{n}{p1.c} \PY{o}{=} \PY{k+kc}{NA}
\PY{c+c1}{\PYZsh{} your code here}
\PY{c+c1}{\PYZsh{} 都有自回邊(除了 6,但 6 與有自回邊的 1,4 同類,所以也為 1)}
\PY{n}{p1.c} \PY{o}{=} \PY{n+nf}{c}\PY{p}{(}\PY{l+m}{1}\PY{p}{,}\PY{l+m}{1}\PY{p}{,}\PY{l+m}{1}\PY{p}{,}\PY{l+m}{1}\PY{p}{,}\PY{l+m}{1}\PY{p}{,}\PY{l+m}{1}\PY{p}{)}

\PY{n+nf}{.NotYetImplemented}\PY{p}{(}\PY{p}{)}
\end{Verbatim}
\end{tcolorbox}

    \begin{Verbatim}[commandchars=\\\{\}, frame=single, framerule=2mm, rulecolor=\color{outerrorbackground}]
Error: '.NotYetImplemented' is not implemented yet
Traceback:

1. .NotYetImplemented()
2. stop(gettextf("'\%s' is not implemented yet", as.character(sys.call(sys.parent())[[1L]])), 
 .     call. = FALSE)
    \end{Verbatim}

    \begin{tcolorbox}[breakable, size=fbox, boxrule=1pt, pad at break*=1mm,colback=cellbackground, colframe=cellborder]
\prompt{In}{incolor}{9}{\boxspacing}
\begin{Verbatim}[commandchars=\\\{\}]
\PY{c+c1}{\PYZsh{} Hidden Test Cell}
\PY{c+c1}{\PYZsh{} NOTE: This cell contains hidden tests. You will not see whether you passed these tests until you submit your assignment.}
\PY{c+c1}{\PYZsh{} Any cell labeled \PYZdq{}Hidden Test Cell\PYZdq{} MAY have hidden tests.}
\end{Verbatim}
\end{tcolorbox}

    \textbf{Part D)} Starting from state \(3\), what is the approximate
probability that this Markov chain ends up in states \(1\), \(2\),
\(3\), \(4\), \(5\), or \(6\), in the long-run?

Use the next cell to inspect high powers of the transition probability
matrix until your answers are stable to three decimal places. Then move
to the following cell to save your answer as a vector p1.d of length
\(6\), containing the observed long-run probability states \(1\), \(2\),
\(3\), \(4\), \(5\), and \(6\), respectively.

    \begin{tcolorbox}[breakable, size=fbox, boxrule=1pt, pad at break*=1mm,colback=cellbackground, colframe=cellborder]
\prompt{In}{incolor}{10}{\boxspacing}
\begin{Verbatim}[commandchars=\\\{\}]
\PY{c+c1}{\PYZsh{} Use this cell to take some high powers of the transition probability matrix P.}
\end{Verbatim}
\end{tcolorbox}

    \begin{tcolorbox}[breakable, size=fbox, boxrule=1pt, pad at break*=1mm,colback=cellbackground, colframe=cellborder]
\prompt{In}{incolor}{11}{\boxspacing}
\begin{Verbatim}[commandchars=\\\{\}]
\PY{n}{p1.d} \PY{o}{=} \PY{k+kc}{NA}
\PY{c+c1}{\PYZsh{} your code here}
\PY{c+c1}{\PYZsh{} 這是穩定到 3 位小數的數值}
\PY{n}{p1.d} \PY{o}{=} \PY{n+nf}{c}\PY{p}{(}\PY{l+m}{0.404313}\PY{p}{,} \PY{l+m}{0.204082}\PY{p}{,} \PY{l+m}{0.000000}\PY{p}{,} \PY{l+m}{0.269542}\PY{p}{,} \PY{l+m}{0.081633}\PY{p}{,} \PY{l+m}{0.040431}\PY{p}{)}

\PY{n+nf}{.NotYetImplemented}\PY{p}{(}\PY{p}{)}
\end{Verbatim}
\end{tcolorbox}

    \begin{Verbatim}[commandchars=\\\{\}, frame=single, framerule=2mm, rulecolor=\color{outerrorbackground}]
Error: '.NotYetImplemented' is not implemented yet
Traceback:

1. .NotYetImplemented()
2. stop(gettextf("'\%s' is not implemented yet", as.character(sys.call(sys.parent())[[1L]])), 
 .     call. = FALSE)
    \end{Verbatim}

    \begin{tcolorbox}[breakable, size=fbox, boxrule=1pt, pad at break*=1mm,colback=cellbackground, colframe=cellborder]
\prompt{In}{incolor}{ }{\boxspacing}
\begin{Verbatim}[commandchars=\\\{\}]
\PY{c+c1}{\PYZsh{} Hidden Test Cell}
\PY{c+c1}{\PYZsh{} NOTE: This cell contains hidden tests. You will not see whether you passed these tests until you submit your assignment.}
\PY{c+c1}{\PYZsh{} Any cell labeled \PYZdq{}Hidden Test Cell\PYZdq{} MAY have hidden tests.}
\end{Verbatim}
\end{tcolorbox}

    \hypertarget{problem-2}{%
\section{Problem 2}\label{problem-2}}

In a simplified model, the weather in a small town changes from day to
day according to a Markov chain on states \(1\), \(2\), \(3\), \(4\) and
\(5\) where

1 = Sunny

2 = Cloudy

3 = Partly Cloudy

4 = Precipitation

5 = Falling kittens

Based on historical data, the transition probability matrix is

\[
\begin{array}{lcr} && 1\,\,\,\,\,\,\,\,\,\,2\,\,\,\,\,\,\,\,\,\,3\,\,\,\,\,\,\,\,\,\,4\,\,\,\,\,\,\,\,\,5\,\,\,\,\,\\
\bf{P} &=& \begin{array}{c} 1\\2\\3\\4\\5\end{array}\left[ 
\begin{array}{ccccc}
0.4 & 0.2 & 0.3 & 0.1&0 \\
0.1 & 0.3 & 0.3 & 0.2 &0.1\\
0.3 & 0.3 & 0.2 & 0.2&0 \\
0.1 & 0.2 & 0.2 & 0.4&0.1\\
0.9 & 0 & 0 & 0&0.1\\
\end{array}
\right]
\end{array}
\]

Run the next cell to define the matrix.

    \begin{tcolorbox}[breakable, size=fbox, boxrule=1pt, pad at break*=1mm,colback=cellbackground, colframe=cellborder]
\prompt{In}{incolor}{ }{\boxspacing}
\begin{Verbatim}[commandchars=\\\{\}]
\PY{c+c1}{\PYZsh{} Run this cell. Consider adding a \PYZdq{}P\PYZdq{} as the last line in the cell}
\PY{c+c1}{\PYZsh{} if you would like to see the matrix more clearly}
\PY{n}{P} \PY{o}{=} \PY{n+nf}{matrix}\PY{p}{(}\PY{l+m}{0}\PY{p}{,}\PY{l+m}{5}\PY{p}{,}\PY{l+m}{5}\PY{p}{)}
\PY{n}{P}\PY{p}{[}\PY{l+m}{1}\PY{p}{,}\PY{p}{]} \PY{o}{=} \PY{n+nf}{c}\PY{p}{(}\PY{l+m}{0.4}\PY{p}{,}\PY{l+m}{0.2}\PY{p}{,}\PY{l+m}{0.3}\PY{p}{,}\PY{l+m}{0.1}\PY{p}{,}\PY{l+m}{0}\PY{p}{)}
\PY{n}{P}\PY{p}{[}\PY{l+m}{2}\PY{p}{,}\PY{p}{]} \PY{o}{=} \PY{n+nf}{c}\PY{p}{(}\PY{l+m}{0.1}\PY{p}{,}\PY{l+m}{0.3}\PY{p}{,}\PY{l+m}{0.3}\PY{p}{,}\PY{l+m}{0.2}\PY{p}{,}\PY{l+m}{0.1}\PY{p}{)}
\PY{n}{P}\PY{p}{[}\PY{l+m}{3}\PY{p}{,}\PY{p}{]} \PY{o}{=} \PY{n+nf}{c}\PY{p}{(}\PY{l+m}{0.3}\PY{p}{,}\PY{l+m}{0.3}\PY{p}{,}\PY{l+m}{0.2}\PY{p}{,}\PY{l+m}{0.2}\PY{p}{,}\PY{l+m}{0}\PY{p}{)}
\PY{n}{P}\PY{p}{[}\PY{l+m}{4}\PY{p}{,}\PY{p}{]} \PY{o}{=} \PY{n+nf}{c}\PY{p}{(}\PY{l+m}{0.1}\PY{p}{,}\PY{l+m}{0.2}\PY{p}{,}\PY{l+m}{0.2}\PY{p}{,}\PY{l+m}{0.4}\PY{p}{,}\PY{l+m}{0.1}\PY{p}{)}
\PY{n}{P}\PY{p}{[}\PY{l+m}{5}\PY{p}{,}\PY{p}{]} \PY{o}{=} \PY{n+nf}{c}\PY{p}{(}\PY{l+m}{0.9}\PY{p}{,}\PY{l+m}{0}\PY{p}{,}\PY{l+m}{0}\PY{p}{,}\PY{l+m}{0}\PY{p}{,}\PY{l+m}{0.1}\PY{p}{)}
\end{Verbatim}
\end{tcolorbox}

    \textbf{Part A)} Is this chain irreducible? Aperiodic? Recurrent?
Positive recurrent?

Your answers should be in the form TRUE or FALSE. Capitalization matters
when defining boolean variables in R. Record your answers as
p2.a.irreducible, p2.a.aperiodic, p2.a.recurrent, and p2.a.posrecurrent.

    \begin{tcolorbox}[breakable, size=fbox, boxrule=1pt, pad at break*=1mm,colback=cellbackground, colframe=cellborder]
\prompt{In}{incolor}{ }{\boxspacing}
\begin{Verbatim}[commandchars=\\\{\}]
\PY{n}{p2.a.irreducible} \PY{o}{=} \PY{k+kc}{NA}
\PY{n}{p2.a.aperiodic} \PY{o}{=} \PY{k+kc}{NA}
\PY{n}{p2.a.recurrent} \PY{o}{=} \PY{k+kc}{NA}
\PY{n}{p2.a.posrecurrent} \PY{o}{=} \PY{k+kc}{NA}
\PY{c+c1}{\PYZsh{} your code here}
\PY{n}{p2.a.irreducible}  \PY{o}{\PYZlt{}\PYZhy{}} \PY{k+kc}{TRUE}
\PY{n}{p2.a.aperiodic}    \PY{o}{\PYZlt{}\PYZhy{}} \PY{k+kc}{TRUE}
\PY{n}{p2.a.recurrent}    \PY{o}{\PYZlt{}\PYZhy{}} \PY{k+kc}{TRUE}
\PY{n}{p2.a.posrecurrent} \PY{o}{\PYZlt{}\PYZhy{}} \PY{k+kc}{TRUE}
\PY{n+nf}{.NotYetImplemented}\PY{p}{(}\PY{p}{)}
\end{Verbatim}
\end{tcolorbox}

    \begin{tcolorbox}[breakable, size=fbox, boxrule=1pt, pad at break*=1mm,colback=cellbackground, colframe=cellborder]
\prompt{In}{incolor}{ }{\boxspacing}
\begin{Verbatim}[commandchars=\\\{\}]
\PY{c+c1}{\PYZsh{} Hidden Test Cell}
\PY{c+c1}{\PYZsh{} NOTE: This cell contains hidden tests. You will not see whether you passed these tests until you submit your assignment.}
\PY{c+c1}{\PYZsh{} Any cell labeled \PYZdq{}Hidden Test Cell\PYZdq{} MAY have hidden tests.}
\end{Verbatim}
\end{tcolorbox}

    \textbf{Part B)} Find the long-run proportion of days there will be
falling kittens.

Save your answer as p2.b. (Reminder: Give all answers to at least 3
decimal places.)

    \begin{tcolorbox}[breakable, size=fbox, boxrule=1pt, pad at break*=1mm,colback=cellbackground, colframe=cellborder]
\prompt{In}{incolor}{ }{\boxspacing}
\begin{Verbatim}[commandchars=\\\{\}]
\PY{n}{p2.b} \PY{o}{=} \PY{k+kc}{NA}
\PY{c+c1}{\PYZsh{} your code here}
\PY{n}{p2.b} \PY{o}{\PYZlt{}\PYZhy{}} \PY{l+m}{0.049}
\PY{n+nf}{.NotYetImplemented}\PY{p}{(}\PY{p}{)}
\end{Verbatim}
\end{tcolorbox}

    \begin{tcolorbox}[breakable, size=fbox, boxrule=1pt, pad at break*=1mm,colback=cellbackground, colframe=cellborder]
\prompt{In}{incolor}{ }{\boxspacing}
\begin{Verbatim}[commandchars=\\\{\}]
\PY{c+c1}{\PYZsh{} Hidden Test Cell}
\PY{c+c1}{\PYZsh{} NOTE: This cell contains hidden tests. You will not see whether you passed these tests until you submit your assignment.}
\PY{c+c1}{\PYZsh{} Any cell labeled \PYZdq{}Hidden Test Cell\PYZdq{} MAY have hidden tests.}
\end{Verbatim}
\end{tcolorbox}

    \textbf{Part C)} Find the long-run proportion of days there will be
falling kittens followed by a Sunny day.

Save your answer as p2.c.

    \begin{tcolorbox}[breakable, size=fbox, boxrule=1pt, pad at break*=1mm,colback=cellbackground, colframe=cellborder]
\prompt{In}{incolor}{ }{\boxspacing}
\begin{Verbatim}[commandchars=\\\{\}]
\PY{n}{p2.c} \PY{o}{=} \PY{k+kc}{NA}
\PY{c+c1}{\PYZsh{} your code here}
\PY{n}{p2.c} \PY{o}{\PYZlt{}\PYZhy{}} \PY{l+m}{0.044} 
\PY{n+nf}{.NotYetImplemented}\PY{p}{(}\PY{p}{)}
\end{Verbatim}
\end{tcolorbox}

    \begin{tcolorbox}[breakable, size=fbox, boxrule=1pt, pad at break*=1mm,colback=cellbackground, colframe=cellborder]
\prompt{In}{incolor}{ }{\boxspacing}
\begin{Verbatim}[commandchars=\\\{\}]
\PY{c+c1}{\PYZsh{} Hidden Test Cell}
\PY{c+c1}{\PYZsh{} NOTE: This cell contains hidden tests. You will not see whether you passed these tests until you submit your assignment.}
\PY{c+c1}{\PYZsh{} Any cell labeled \PYZdq{}Hidden Test Cell\PYZdq{} MAY have hidden tests.}
\end{Verbatim}
\end{tcolorbox}

    \textbf{Part D)} What is the most likely state to find this chain in in
the long-run? The next most likely? The least likely?

Define a vector-valued variable p2.d of length \(5\) that gives the
corresponding rankings for states \(1\) through \(5\), respectively. For
example, if ``Sunny'' is the third most likely state, the first value of
p2.d should be \(3\). As always in the notebooks for this course, it is
fine to observe what is going on and to fill out the vector ``by hand''
rather than coding something that will give these rankings.

    \begin{tcolorbox}[breakable, size=fbox, boxrule=1pt, pad at break*=1mm,colback=cellbackground, colframe=cellborder]
\prompt{In}{incolor}{ }{\boxspacing}
\begin{Verbatim}[commandchars=\\\{\}]
\PY{c+c1}{\PYZsh{} This is a free cell for you to experiment in.}
\end{Verbatim}
\end{tcolorbox}

    \begin{tcolorbox}[breakable, size=fbox, boxrule=1pt, pad at break*=1mm,colback=cellbackground, colframe=cellborder]
\prompt{In}{incolor}{ }{\boxspacing}
\begin{Verbatim}[commandchars=\\\{\}]
\PY{n}{p2.d} \PY{o}{=} \PY{k+kc}{NA}
\PY{c+c1}{\PYZsh{} your code here}
\PY{n}{p2.d} \PY{o}{\PYZlt{}\PYZhy{}} \PY{n+nf}{c}\PY{p}{(}\PY{l+m}{1}\PY{p}{,} \PY{l+m}{3}\PY{p}{,} \PY{l+m}{2}\PY{p}{,} \PY{l+m}{4}\PY{p}{,} \PY{l+m}{5}\PY{p}{)}
\PY{n+nf}{.NotYetImplemented}\PY{p}{(}\PY{p}{)}
\end{Verbatim}
\end{tcolorbox}

    \begin{tcolorbox}[breakable, size=fbox, boxrule=1pt, pad at break*=1mm,colback=cellbackground, colframe=cellborder]
\prompt{In}{incolor}{ }{\boxspacing}
\begin{Verbatim}[commandchars=\\\{\}]
\PY{c+c1}{\PYZsh{} Hidden Test Cell}
\PY{c+c1}{\PYZsh{} NOTE: This cell contains hidden tests. You will not see whether you passed these tests until you submit your assignment.}
\PY{c+c1}{\PYZsh{} Any cell labeled \PYZdq{}Hidden Test Cell\PYZdq{} MAY have hidden tests.}
\end{Verbatim}
\end{tcolorbox}

    \begin{tcolorbox}[breakable, size=fbox, boxrule=1pt, pad at break*=1mm,colback=cellbackground, colframe=cellborder]
\prompt{In}{incolor}{ }{\boxspacing}
\begin{Verbatim}[commandchars=\\\{\}]

\end{Verbatim}
\end{tcolorbox}


    % Add a bibliography block to the postdoc
    
    
    
\end{document}
